%konfiguracja bez treści, aby podłączać do innych plików

\usepackage[OT4,plmath,MeX]{polski}


\usepackage{indentfirst} % polski zwyczaj dla wciec akapitowych
\usepackage{graphicx}
%\usepackage[decimalsymbol=comma]{siunitx}
\usepackage{url}
\usepackage{pdflscape} 
\usepackage{subfigure}

\usepackage{wrapfig}
\usepackage[top=2cm, bottom=2cm, left=2.25cm, right=2.25cm]{geometry}

%\oddsidemargin 0.5cm
%\evensidemargin -0.5cm
\usepackage{algorithmic}
\usepackage[Algorytm]{algorithm}
\usepackage{listings}
\renewcommand{\labelitemi}{$\circ$}
\renewcommand{\today}{\oldstylenums{\number\day}~\ifcase\month\or
  stycznia\or lutego\or marca\or kwietnia\or maja\or czerwca\or
  lipca\or sierpnia\or września\or października\or
  listopada\or grudnia\fi \space\oldstylenums{\number\year} r.}

\usepackage{caption}
\usepackage{color}
\usepackage{rotating}
\usepackage{floatflt}
\usepackage{float}
\newfloat{scheme}{pb}{htbp} % second argument was {plt}
\floatname{scheme}{Schemat}
\newfloat{plot}{pb}{htbp} % second argument was {plt}
\floatname{plot}{Wykres}
\renewcommand*{\figurename}{Rysunek}
\renewcommand*{\tablename}{Tabela} 
\renewcommand*{\lstlistingname}{Kod}
\captionsetup{labelsep=period}
\usepackage{lscape}

\usepackage{amsmath}
\usepackage{mathtools}

\definecolor{dkgreen}{rgb}{0,0.5,0}
\definecolor{dkblue}{rgb}{0,0,0.7}
\definecolor{gray}{rgb}{0.5,0.5,0.5}
\definecolor{ltgray}{rgb}{0.8,0.8,0.8}
\definecolor{mauve}{rgb}{0.58,0,0.82}
\definecolor{maroon}{rgb}{0.5,0,0}

\newcommand{\HRule}{\rule{\linewidth}{0.5mm}}
\newcommand{\linia}{\rule{\linewidth}{0.1mm}}

\usepackage[ 
	pdftitle=\MYPDFTITLE,
	pdfauthor=\MYPDFAUTHOR,
	bookmarks=true, 
	bookmarksnumbered=true, 
	unicode=true, 
	pdftex, 
	pdfnewwindow=true,
	colorlinks=true,
	linkcolor=blue,
	hidelinks
]{hyperref} 


\lstset{
	language=\MYLSTSETLANGUAGE,
	frame=\MYLSTSETFRAME,
	morekeywords=\MYLSTSETKEYWORDS,
	basicstyle=\footnotesize,
	numbers=left, 
	numberstyle=\tiny\color{gray},
	stepnumber=1,
	columns=fixed,
	numbersep=20pt, 
	backgroundcolor=\color{white}, 
	showspaces=false, 
	showstringspaces=true, 
	showtabs=false, 
	rulecolor=\color{ltgray},
	tabsize=4,
	breakatwhitespace=true,
	breakindent=30pt,
	breaklines=true,
	breakautoindent=true,
	prebreak=\mbox{{\color{red}$\hookleftarrow$}},
	postbreak=\mbox{{\color{red}$\hookrightarrow$}}\space,
	keywordstyle=\color{dkblue},
	commentstyle=\color{red},
	stringstyle=\color{dkgreen},
	identifierstyle=\color{maroon},
	morekeywords=_MY_MACRO_LSTSET_KEYWORDS
}


%\renewcommand{\thefootnote}{_MY_MACRO_FOOTNOTE}
%define _MY_MACRO_FOOTNOTE $\dagger$
						  %\fnsymbol{footnote}
						  % $\bullet${}
						  %\Roman{footnote}
						  %\alph{footnote}


\usepackage{float}
%\newfloat{schemat}{pb}{htpb} % second argument should be {htbp} in your actual iedocument
%\floatname{schemat}{Schemat}
\usepackage{pdfpages}

%\newfloat{wykres}{pb}{htpb} % second argument should be {htbp} in your actual document
%\floatname{wykres}{Rys.}

\usepackage{tocbasic}
\DeclareNewTOC[%
  type=schemat,%
  types=schems,% used in the \listof.. command
  float,% define a floating environment
  floattype=4,% see below
  name=Schemat,%
  listname={Spis schematów}%
]{lop}

\DeclareNewTOC[%
  type=wykres,%
  types=wykress,% used in the \listof.. command
  float,% define a floating environment
  floattype=4,% see below
  name=Rys.,%
  listname={Spis rysunków}%
]{lor}



\newcommand{\itab}[1]{\hspace{0em}\rlap{#1}}
\newcommand{\tab}[1]{\hspace{.29\textwidth}\rlap{#1}}


\renewcommand\algorithmicforall{\textbf{Dla wszystkich}}
\renewcommand\algorithmicfor{\textbf{Dla}}
\renewcommand\algorithmicdo{\textbf{wykonaj:}}
\renewcommand\algorithmicendfor{\textbf{Koniec.}}
\renewcommand\algorithmicif{\textbf{Jeżeli}}
\renewcommand\algorithmicthen{\textbf{to:}}
\renewcommand\algorithmicendif{\textbf{Koniec.}}
\renewcommand\algorithmicelse{\textbf{W przeciwnym wypadku:}}
\renewcommand\algorithmicwhile{\textbf{Dopóki}}
\renewcommand\algorithmicendwhile{\textbf{Koniec.}}
\renewcommand\algorithmicreturn{\textbf{Zwróć}}
\setlength{\algorithmicindent}{2em}
